\pagestyle{empty}

\usepackage{float,graphicx,parskip,tabularx,url}
\usepackage[utf8]{inputenc}
\usepackage[version=4]{mhchem}
\usepackage{caption}
\usepackage{subcaption}
\usepackage[T1]{fontenc}
\usepackage{amsmath, amsthm, amssymb, amsfonts}
\usepackage{thmtools}
\usepackage{graphicx}
\usepackage{setspace}
\usepackage[a4paper,top=25mm,bottom=25mm,inner=25mm,outer=25mm]{geometry}
\usepackage{float}
\usepackage{hyperref}
\usepackage[french]{babel}
\usepackage{framed}
\usepackage[dvipsnames]{xcolor}
\usepackage{tcolorbox}
\usepackage{exercise}
\def\ExerciseName{Exercice}
\usepackage{mdframed}

\colorlet{LightGray}{Gray!8}
\colorlet{LightOrange}{Orange!10}
\colorlet{LightGreen}{Green!8}
\colorlet{LightBlue}{Blue!8}
\colorlet{LightRed}{Red!15}
\colorlet{Yellow}{Yellow!15}

\newcommand{\HRule}[1]{\rule{\linewidth}{#1}}

\newmdenv[
  leftmargin = 0.9em,
  innerleftmargin = 0.5em,
  innertopmargin = 0.5em,
  innerbottommargin = 0.5em,
  innerrightmargin = 0pt,
  rightmargin = 0pt,
  linewidth = 0.2pt,
  topline = false,
  rightline = false,
  bottomline = true,
  leftline = true,
]{leftbottombarenv}

\declaretheoremstyle[
  postheadhook=\leavevmode\begin{leftbottombarenv},
    prefoothook=\end{leftbottombarenv},
]{leftbottombarsty}

\newmdenv[
  leftmargin = 0.9em,
  innerleftmargin = 0.5em,
  innertopmargin = 0.5em,
  innerbottommargin = 0.5em,
  innerrightmargin = 0pt,
  rightmargin = 0pt,
  linewidth = 0.2pt,
  topline = false,
  rightline = false,
  bottomline = false,
  leftline = true,
]{leftbarenv}

\declaretheoremstyle[
  postheadhook=\leavevmode\begin{leftbarenv},
    prefoothook=\end{leftbarenv},
]{leftbarsty}

\declaretheoremstyle[name=Théorème]{thmsty}
\declaretheorem[style=thmsty,numberwithin=section]{theorem}
\tcolorboxenvironment{theorem}{colback=LightOrange, boxrule=1pt, arc=2pt}

\declaretheoremstyle[name=Corollaire,]{corsty}
\declaretheorem[style=corsty,numberwithin=section]{corollary}
\tcolorboxenvironment{corollary}{colback=LightRed, boxrule=1pt, arc=2pt}

\declaretheoremstyle[name=Lemme,]{lemsty}
\declaretheorem[style=lemsty,numberwithin=section]{lemma}
\tcolorboxenvironment{lemma}{colback=Yellow, boxrule=1pt, arc=2pt}

\declaretheoremstyle[name=Proposition,]{prosty}
\declaretheorem[style=prosty,numberwithin=section]{proposition}
\tcolorboxenvironment{proposition}{colback=LightBlue, boxrule=1pt, arc=2pt}

\declaretheoremstyle[name=Propriété,]{propristy}
\declaretheorem[style=propristy,numberwithin=section]{propriety}
\tcolorboxenvironment{propriety}{colback=LightGray, boxrule=1pt, arc=2pt}

\declaretheoremstyle[name=Définition,]{defsty}
\declaretheorem[style=defsty,numberwithin=section]{definition}
\tcolorboxenvironment{definition}{colback=LightGreen, boxrule=1pt, arc=2pt}

\declaretheorem[style=leftbarsty,name=Exemple,numberwithin=section]{example}

\declaretheorem[style=leftbottombarsty,name=Remarque,numberwithin=section]{remark}

\setcounter{tocdepth}{3}
\setcounter{secnumdepth}{3}

\usepackage{fancyhdr}
\pagestyle{fancy}
\fancyhf{}
\fancyhead[R]{\leftmark}
\fancyfoot[C]{\thepage}

\fancypagestyle{plain}{
  \renewcommand{\headrulewidth}{0pt}
  \fancyhf{}
  \fancyfoot[C]{\thepage}
}

\renewcommand{\footrulewidth}{0pt}

\usepackage{enumitem}
\setlist[itemize]{label={\textbullet}}

\usepackage{bbold}
\usepackage[bottom]{footmisc}
\usepackage{collectbox}
\usepackage{import}
\newcommand{\incsvg}[3][1.]{%
  \def\svgwidth{#1\columnwidth}
  \graphicspath{{#2}}
  \input{#3.pdf_tex}
}

% Spaced arrays
\renewcommand{\arraystretch}{1.5}

% Circling text
\usepackage{tikz}
\newcommand*\circled[1]{\tikz[baseline=(char.base)]{
    \node[shape=circle,draw,inner sep=1pt] (char) {#1};}}
\newcommand{\cplus}{\tikz[baseline=(char.base)]{
    \node[shape=circle,draw,inner sep=1pt] (char) {+};}\space}
\newcommand{\cminus}{\tikz[baseline=(char.base)]{
    \node[shape=circle,draw,inner sep=3.2pt] (char) {-};}\space}

% Nice commands
\newcommand{\R}{\mathbb{R}}
\newcommand{\N}{\mathbb{N}}
\newcommand{\Q}{\mathbb{Q}}
\newcommand{\C}{\mathbb{C}}
\newcommand{\Z}{\mathbb{Z}}
\newcommand{\independant}{\perp\!\!\!\!\perp}
\makeatletter
\newcommand{\mybox}{%
  \collectbox{%
    \setlength{\fboxsep}{3pt}%
    \fbox{\BOXCONTENT}%
  }%
}
\makeatother

% Sans-serifs maths
\usepackage[cm]{sfmath}

% Better paragraph and subparagraph formatting
\usepackage[raggedright]{titlesec}
\usepackage{blindtext}
\usepackage{ulem}
\titleformat{\paragraph}[hang]{\normalfont\normalsize\bfseries}{\theparagraph}{0.5em}{}
\titleformat{\subparagraph}[hang]{\normalfont\normalsize}{\theparagraph}{0.5em}{\uline}

\usepackage{fontawesome}