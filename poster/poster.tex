% Unofficial University of Cambridge Poster Template
% https://github.com/andiac/gemini-cam
% a fork of https://github.com/anishathalye/gemini
% also refer to https://github.com/k4rtik/uchicago-poster

\documentclass[final]{beamer}

% ====================
% Packages
% ====================

\usepackage[T1]{fontenc}
\usepackage{lmodern}
\usepackage[orientation=portrait,size=a0,scale=1.0]{beamerposter}
\usetheme{gemini}
\usecolortheme{nott}
\usepackage{graphicx}
\graphicspath{{figures/}}
\usepackage{float}
\usepackage{subfloat}
\usepackage{caption}
\usepackage{subcaption}
\usepackage{booktabs}
\usepackage{tikz}
\usetikzlibrary{bayesnet}
\usetikzlibrary{arrows}
\usepackage{pgfplots}
\pgfplotsset{compat=1.18}
\usepackage{anyfontsize}
\usepackage{multirow}
\usepackage{stmaryrd}

% ====================
% Lengths
% ====================

\newlength{\sepwidth}
\newlength{\colwidth}
\setlength{\sepwidth}{0.025\paperwidth}
\setlength{\colwidth}{0.45\paperwidth}

% ====================
% Macros
% ====================

\newcommand{\fracpartial}[2]{\frac{\partial #1}{\partial  #2}}
\newcommand{\prob}{\mathbb{P}}
\newcommand{\R}{\mathbb{R}}
\newcommand{\N}{\mathbb{N}}
\newcommand{\E}{\mathbb{E}}
\renewcommand{\O}{\mathcal{O}}
\renewcommand{\L}{\mathcal{L}}
\newcommand{\eps}{\varepsilon}
\newcommand{\dd}{\, \mathrm{d}}
\newcommand{\J}{\mathcal{J}}
\DeclareMathOperator*{\argmin}{arg\,min}
\DeclareMathOperator*{\argmax}{arg\,max}
\DeclareMathOperator*{\minimize}{minimize}
\DeclareMathOperator*{\KL}{KL}
\newcommand{\intset}[2]{\{#1, ..., #2\}}
\newcommand{\mnbs}{\nobreak\hspace{.16667em}}
\newcommand{\jump}{\newline\newline}
\newcommand{\separatorcolumn}{\begin{column}{\sepwidth}\end{column}}

% ====================
% Header, footer, etc.
% ====================

\title{Mixture Models for Graph Clustering}

\author{Sofiane Ezzehi \inst{1} \and Bastien Le Chenadec \inst{1} \and Theïlo Terrisse \inst{1}}

\institute[shortinst]{École des Ponts ParisTech}

\footercontent{
  \href{https://github.com/bastienlc/mmrg}{https://github.com/bastienlc/mmrg} \hfill Probabilistic Graphical Models
   \hfill
  Master Mathématiques Vision Apprentissage}

\logoright{\includegraphics[height=4cm]{figures/logo_mva.png}}
\logoleft{\includegraphics[height=4cm]{figures/logo_ens.png}}

% ====================
% Body
% ====================

\begin{document}

\begin{frame}[t]
  \begin{columns}[t]
    \separatorcolumn

    \begin{column}{\colwidth}

      \begin{block}{Context}
      \end{block}

      \begin{block}{The Stochastic Block Model}
        \begin{column}{0.5\colwidth}
          \justifying
          The \textbf{Stochastic Block Model} is a \textit{mixture} model for graphs. Each node is assigned to a \textit{class}, and conditionally on these classes the probability of an edge between two nodes $i$ and $j$ only depends on their class memberships. This model draws inspiration from mixture models for distribution of degrees, and the Erdös-Rényi model which deals with the probability for two given nodes of being connected.
        \end{column}
        \begin{column}{0.5\colwidth}
          \begin{figure}[H]
            \centering
            \tikz{
              \node[obs] (X_{ij}) {$X_{ij}$};
              \node[latent,left=of X_{ij}, xshift=-3cm, label={[name=label1,text height=1.2em]below:$i=1,\dots,n$}] (Z_i) {$Z_i$};
              \node[latent,right=of X_{ij}, xshift=3cm, label={[name=label2,text height=1.2em]below:$j=i+1,\dots,n$}] (Z_j) {$Z_j$};
              \node[const, above=of X_{ij}, yshift=0cm](alpha){$\alpha$};
              \node[const, below=of X_{ij}, yshift=-2cm](pi){$\pi$};
              \plate [inner sep=.5cm] {} {(Z_i)(X_{ij})(Z_j)(label1)(label2)} {};
              \plate [inner sep=.25cm] {} {(Z_j)(X_{ij})(label2)} {};
              \edge {Z_i,Z_j} {X_{ij}}
              \edge {alpha} {Z_i}
              \edge {alpha} {Z_j}
              \edge {pi} {X_{ij}}
            }
            \caption{Graphical model of the SBM}
            \label{fig:graphical_model}
          \end{figure}
        \end{column}

        Formally, we consider an undirected graph with $n$ nodes and no self-loops. We denote $X$ the adjacency matrix of this graph, thus $X_{ij}\in \{0,1\}$ denotes the existence of an edge between nodes $i$ and $j$.

        The model is parametrized by $Q$ the number of classes, $\alpha\in [0,1]^Q$ the prior distribution on the classes, and $\pi\in [0,1]^{Q\times Q}$ the probability of an edge between two nodes of different classes. We also introduce the random variables $Z_i\in \{0,1\}^Q$ for $i\in \llbracket 1,n \rrbracket$, that represent the membership of node $i$ to each class. The prior distributions on $Z$ and $X$ are given by :
        \begin{equation}
          \begin{cases}
            \forall i\in \llbracket 1,n \rrbracket, \quad \sum_{q=1}^Q Z_{iq} = 1 \quad\text{(unique class)}            \\
            \forall q\in \llbracket 1,Q \rrbracket,\quad \mathbb{P}(Z_{iq}=1)=\alpha_q \quad\text{(class distribution)} \\
            \forall q,l\in \llbracket 1,Q \rrbracket, \quad \forall i\neq j\in \llbracket 1,n \rrbracket, \quad \mathbb{P}(X_{ij}=1\,|\,Z_{iq}=1,Z_{jl}=1)=\pi_{ql} \quad\text{(edge probability)}
          \end{cases}
        \end{equation}
      \end{block}

      \begin{alertblock}{The variational Expectation-Maximization algorithm}
      \end{alertblock}

      \begin{block}{Variants}

      \end{block}

    \end{column}

    \separatorcolumn

    \begin{column}{\colwidth}

      \begin{block}{Experiments on an SBM dataset}
      \end{block}

      \begin{block}{Second set of experiments}
      \end{block}

      \begin{block}{References}
        \nocite{*}
        \footnotesize{\bibliographystyle{plain}\bibliography{poster}}
      \end{block}

    \end{column}

    \separatorcolumn

  \end{columns}
\end{frame}

\end{document}
