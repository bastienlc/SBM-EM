\section{Ce que j'ai compris}

Un modèle de graphes  aléatoires qui à la fois :
\begin{itemize}
    \item est un modèle de mélange
    \item décrit explicitement les liens entre les noeuds
\end{itemize}

\paragraph{Notations :}
\begin{itemize}
    \item $X_{ij}\in \{0,1\}$ l'existence d'une arête entre les noeuds $i$ et $j$.
    \item $Q$ le nombre de classes, avec $\{\alpha_1, \dots, \alpha_Q\}$ les proportions de chaque classe.
    \item $Z_{iq} \in \{0,1\}$ l'appartenance du noeud $i$ à la classe $q$.
    \item $\pi_{ql}=\pi_{lq}$ la probabilité d'un lien entre un noeud de la classe $q$ et un noeud de la classe $l$.
\end{itemize}

\paragraph{Hypothèses :}
\begin{itemize}
    \item Bonne définition des classes : $\sum_q Z_{iq}=1$, $\sum_q \alpha_q=1$ et $\alpha_q=P(Z_{iq}=1)$.
    \item Pas de self loop : $X_{ii}=0$
    \item Indépendances des arêtes conditionnellement aux classes : $X_{ij}| \{i\in q, j\in l\} \sim \mathcal{B}(\pi_{ql})$.
\end{itemize}

\paragraph{Propriétés importantes :}
\begin{itemize}
    \item Distribution des degrés binomiale (conditionnellement aux classes).
\end{itemize}

\vspace{5mm}

Extension : modèle indépendant, chaque classe a une propension à se lier avec les autres classes. On a alors $\pi_{ql}=\eta_q\eta_l$.

Log-vraissemblance du modèle (est-ce que le modèle observé est probable au vu des paramètres fixés ?) :
$$\log \mathcal{L}(X, Z) = \sum_i\sum_qZ_{iq}\log \alpha_q + \frac{1}{2} \sum_{i\neq j}\sum_{q,l}Z_{iq}Z_{jl}\log b(X_{ij,\pi_{ql}})$$
avec $b(x,\pi)=\pi^x(1-\pi)^{1-x}$.

Estimation : on veut estimer les paramètres du modèle en maximisant la log-vraisemblance. L'algorithme EM n'est pas utilisable ici. Le papier se concentre sur l'optimisation d'une borne inférieure de la log-vraisemblance, de façon similaire à l'algorithme EM.

\begin{itemize}
    \item Étape E : on calcule les paramètres optimaux $\hat{\tau}_{iq}$ de la distribution conditionnelle approchée (point fixe).
    \item Étape M : on maximise la log-vraisemblance approchée par rapport aux paramètres du modèle.
\end{itemize}
On note que :
\begin{itemize}
    \item On a une garantie de croissance de la log-vraisemblance approchée à chaque itération.
    \item Le nombre de classes est fixé à l'avance. Le papier propose un critère de sélection du nombre de classes.
\end{itemize}


\section{Questions}

Rentrer dans le détail des preuves ?

\section{Objectifs}

\begin{itemize}
    \item Implémenter l'algorithme d'estimation.
    \item Tester l'implémentation sur le même dataset que le papier.
    \item Trouver des datasets pertinents sur lesquels tester l'algorithme.
    \item Proposer une méthode d'estimation alternative (idées, preuves, implémentation)
    \item Investiguer la pertinence du critère de sélection du nombre de classes. Implémenter d'autres critères, montrer leur pertinence en fonction du type de tâche.
    \item Étudier le sampling après estimation : est-ce que le modèle est capable de générer des graphes similaires à ceux du dataset ?
\end{itemize}

\section{Ce qui sera simple}

\begin{itemize}
    \item Sampler un graphe aléatoire avec des paramètres fixés.
    \item Implémenter l'algorithme d'estimation.
\end{itemize}

\section{Ce qui sera difficile}

\begin{itemize}
    \item Faire tourner l'algorithme efficacement.
    \item Trouver des méthodes d'estimation alternatives qui s'appliquent.
\end{itemize}

\section{Idées, critiques}

Comparaison avec d'autres modèles ?

\section{Implémentation}

\subparagraph*{E-step :}

On a :
\begin{equation}
    \hat{\tau}_{iq} \propto \alpha_q\prod_{j\neq i}\prod_lb(X_{ij};\pi_{ql})^{\hat{\tau}_{jl}}
\end{equation}
